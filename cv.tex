%!TEX TS-program = xelatex
%!TEX encoding = UTF-8 Unicode
\documentclass[11pt]{article}
\usepackage[a4paper, portrait, margin=1.4cm]{geometry}
\usepackage{fontspec}
\usepackage{ragged2e}
\usepackage[dvipsnames]{xcolor}
\usepackage{nopageno}
\usepackage[ddmmyyyy]{datetime}
\usepackage{multicol}
\usepackage{enumitem}
\usepackage{flafter}
\usepackage{graphicx}
\definecolor{mygreen}{HTML}{3A7F6D}
\graphicspath{ {./images/} }
\setmainfont[Mapping=tex-text,BoldFont=NimbusSanL-Bol.otf,ItalicFont=NimbusSanL-RegIta.otf]{NimbusSanL-Reg.otf}
\begin{document}
\raggedright
\textcolor{mygreen}{{\fontsize{25}{30}\textbf{Morten Jakeobsen}}}\newline
\textcolor{mygreen}{{\fontsize{13}{16}\textbf{Site Reliability Engineer}}}\newline

\textcolor{mygreen}{{\fontsize{13}{16}\textbf{{Profile}}}}\newline

Passionate and creative system administrator with five years of professional experience. During my career I have
built numerous solutions and systems and I have been supporting a growing number of end users over the phone,
email, ticket system and in person. I have built many custom solutions and been involved in several major
projects with things like migrating server rooms and building customer products. I love to optimize workflows
and automate systems to make things easier to use. I do this through simplistic design and robust functionality.
I’m a keen learner, and I like to explore new technology to expand my knowledge.\newline

\textcolor{mygreen}{{\fontsize{13}{0}\textbf{{Skills}}}}
\begin{multicols}{3}
Python\newline
PHP\newline
HTML\newline
CSS\newline
Shell Scripting\newline
MySQL\newline
SQLite\newline
Linux\newline
Git\newline
GitHub\newline
GitLab\newline
Docker\newline
Ansible\newline
phpMyAdmin\newline
AWS (Still learning)\newline
DigitalOcean\newline
JIRA\newline
G-Suite\newline
Wordpress\newline
Joomla\newline
IoT development
\end{multicols}

\textcolor{mygreen}{{\fontsize{13}{16}\textbf{{Experience}}}}\newline

\textbf{2020 - now      Anemo Analytics ApS}\newline
My main focus during this position is to develop, improve and deploy our on-prem analytics solution to customers
around the world. The platform stack consists of ArangoDB, Python 3.8 and Docker. Through the companys
proprietary analytics models we provide customers real time analysis of their parks, and help them improve the
overall performance of their turbines, and through this we can extend the lifespan of their turbines.\newline

\textbf{2014 - 2019     Silicom Denmark A/S} (formerly known as Fiberblaze A/S)\newline
I worked for Silicom as part of my Data Technician education. I was the sole fulltime IT technician during my
employment with Silicom. I managed everything from end user support to infrastructure deployment and
maintenance. I also worked on several internal software projects that helped maintain the business aswell as
proof of contect projects for our customers.\newline

\textcolor{mygreen}{{\fontsize{13}{16}\textbf{{Recommendations}}}}\newline

\textbf{Henrik Lilja}, CEO of Silicom Denmark A/S\newline
I have had the pleasure to work with Morten at Fiberblaze, later Silicom Denmark, for the last four years where
he has been working as a trainee. I have seen Morten successfully take charge of a variety of tasks normally far
exceeding in responsibility level what would be given to a trainee.  Morten has thus been doing design and
coordination tasks needed in a fast-growing company on top of daily support tasks. Morten has also successfully
participated in some solution deliveries to end customers in the automotive business where he put his
programming skills to work. I can thus recommend Morten as a versatile IT professional who can fit into a number
of positions where he can use his keen interest in the latest IT technologies.\newline

\newpage

\textcolor{mygreen}{{\fontsize{13}{16}\textbf{{Software Projects}}}}\newline

\textbf{Four Seven Four}\newline
\textcolor{mygreen}{https://gitlab.com/jakeobsen/foursevenfour/blob/master/474.py}\newline
This is my personal time tracking software, I wrote this program because I needed an efficient way to keep track
of my time at work, the program stores timestamps in an sqlite database and by calculating the deltas between
the stored timestamps, the program can figure out how much time I have spent at work. I also use this program to
figure out if I need time off due to overtime.\newline

\textbf{EBS - External Backup Script}\newline
\textcolor{mygreen}{https://gitlab.com/jakeobsen/ebs/blob/master/ebs}\newline
EBS is designed to synchronizes a folder on an internal disk, with a folder on an external disk. Making it
possible to do automated external rotated backups. I use it with Dirvish, to synchronize the backup structure,
with an external disk. The external disk is rotated on a weekly basis. I initially wrote EBS as a bash shell
script, and then later rewrote it in Python.\newline

\textbf{tempager.py}\newline
\textcolor{mygreen}{https://gitlab.com/jakeobsen/monitoring-plugins/blob/master/tempager.py}\newline
I needed a small plugin that could read the temperature data from a legacy sensor in our datacenter. The sensor
proved to be difficult to parse, as the output from the sensor was in a non compliant JSON format, and the http
service did not send proper response headers. I had to create my own TCP HTTP request using a telnet plugin,
then use regular expressions to fix the response, and make it JSON compliant, before parsing and outputting to
Munin or Nagios.\newline

\textbf{networkscanner}\newline
\textcolor{mygreen}{https://gitlab.com/jakeobsen/networkscanner/blob/master/networkscanner.py}\newline
I needed a tool to scan and collect information about our network and I needed it to be automatic. This tool
scans a a network for hosts using nmap and arpscan, and collects the results into a data object. It then scans
each host to find open ports and updates the data object accordingly. Once done the script will save the result
in either a mysql database or json file and print a report. The tool can be run unattended to update the
database, and a report can be printed from the database without having to scan.\newline

\textcolor{mygreen}{{\fontsize{13}{16}\textbf{{Other Projects}}}}\newline

\textbf{2019      Graduation Ansible Project}\newline
\textcolor{mygreen}{https://gitlab.com/jakeobsen/ansible-svendeprove}\newline
I decided to use Ansible as part of my Data Technician graduation project.  The goal was to build a Wordpress
web hosting company using virtual machines on a VMware based infrastructure. I used  a dedicated virtual machine
as my Ansible server, three web servers and one mysql server. I also had a Synology fileserver which exported an
NFS share that was mounted in the web servers webroot path. The Ansible configuration will deploy the necessary
configuration on all servers and automatically configure everything required.\newline

\textbf{2018      Interviews @ Vi elsker 90'erne (We love the 90’s)}\newline
\textcolor{mygreen}{https://www.facebook.com/vielsker/videos/1898307746886069/}\newline
I was in charge of the video production for the interviews filmed at “Vi Elsker 90'erne Rødovre”. At the same
time I was training a new camera man for our video team. Together with “Lille Lars” we filmed 16 interviews with
various known artists and afterwards I edited the content.\newline

\textbf{2017 - 2018   DJ Skoge og Bassministeriet}\newline
\textcolor{mygreen}{https://www.tv2lorry.dk/artikel/djs-skabte-fest-paa-julemaerkehjem-med-stor-check}\newline
“DJ Skoge og Bassministeriet” was a live DJ show where we played 90's music (and early 2000's) on Facebook and
later on Twitch. On Facebook alone we had around 5.6 million watched minutes during the season and around 2.2
million views. My job on the team were to manage the video setup. I configured our cameras, video broadcasting
software and edited some of our released videos. We also raised 204.501 DKK (about 30.000 USD) for the Christmas
Seal Foundation Home (Julemærkehjem) in Roskilde.\newline

\textcolor{mygreen}{{\fontsize{13}{16}\textbf{{Voluntary work}}}}\newline

\textbf{2010 - 2017   Radio 10FM}\newline
In 2010 I joined Radio 10FM as a radiohost where I made a monthly music program. After seeing the state of the
radio stations IT infrastructure, I started helping out by providing simple support. Later I helped the radio
transition into the internet era by rebuilding the infrastructure using modern technology. And eventually I
assumed full responsibility for the IT infrastructure.\newline

\textbf{2014 - 2016   Free Software Foundation}\newline
I helped out with the weekly online Free Software Directory meeting, where we would review software and their
software licenses, update the Free Software Directory and add new entries. I was also in charge of approving new
submissions from software vendors.\newline

\textbf{2011 - 2014   Radiogruppen}\newline
At the time Radiogruppen consisted of: RadioHLR, GlobalFM, GoldFM and Partyzone.nu
I was a host, technician and
web developer at the stations. I worked on backend systems that allowed the website to communicate with the
radio automation software. I also worked on the FM transmitter sites, by building and supporting the computer
parts of our solutions.\newline

\textbf{2008 - 2012   Dinhost.net}\newline
Dinhost.net was a free web hosting company. From 2008 to 2010 I provided end user support to our growing user
base. After reaching our maximum capacity in 2010 we decided to end our free hosting services and relaunch
Dinhost as a paid web hosting provider. We spent the next year planning and rebuild the company from scratch. In
2015 the company was sold to Meebox.\newline

\textcolor{mygreen}{{\fontsize{13}{16}\textbf{{Education}}}}\newline

\textbf{2014 - 2019}   Infrastructure Data Technician, TEC Ballerup\newline
\textbf{2009}          10th form-level school-leaving examination (of the Danish Folkeskole)\newline
\textbf{2008}          9th form-level school-leaving examination (of the Danish Folkeskole)\newline

\textcolor{mygreen}{{\fontsize{13}{16}\textbf{{Languages}}}}\newline

\textbf{Danish}        Native language, speaks and writes fluently\newline
\textbf{English}       Second language, speaks and writes fluently\newline

\end{document}
